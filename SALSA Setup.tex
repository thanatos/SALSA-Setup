\documentclass{article}
\usepackage[
	pdftex,
	colorlinks=true,urlcolor=blue,
	pdfauthor={Roy Wellington},
	pdftitle={SALSA Setup Guide},
	pdfsubject=SALSA,
	pdfkeywords=SALSA
]{hyperref}
\usepackage{listings}
\usepackage{anysize}
\newcommand{\bslash}{\textbackslash}
\newcommand{\envPATH}{\texttt{PATH}}
\newcommand{\envCLASSPATH}{\texttt{CLASSPATH}}

\title{SALSA Setup Guide}
\begin{document}
\marginsize{1 in}{1 in}{1 in}{1 in}

\section{Settings up JRE \& the JDK}
JRE is the Java Runtime Environment, which you need to run Java applications,
and the JDK, which you need to compile Java applications. For SALSA, you will
need both.

\subsection{Linux}
Get both Java and the JDK through your distribution. Your package manager
(ie, apt-get/Synaptic for Ubuntu/Debian) should have a package for it.

\subsection{Windows}
You probably already have the JRE. If you do, it will be located under
\texttt{C:\bslash Program Files\bslash Java\bslash jre*}. If there are
multiple folders, you probably have multiple versions installed. Choose
the highest version. If you see a folder named \texttt{jre6}, it is not
version 6. To manually see the version of a version of Java, do the
following in a command prompt:
\begin{lstlisting}
C:
cd \"Program Files"\Java\(your Java version's folder)\bin
.\java.exe -version
\end{lstlisting}

If you don't have it, go to \url{http://java.sun.com/javase/downloads/}, and grab "Java SE Runtime Environment" (SE stands for "Standard Edition") and install it.

Once you have the JRE, get the JDK too. Much the same procedure.

\subsubsection{Add Java to your Path}
Add Java to your \envPATH. You need both \texttt{java} and \texttt{javac}
in your \envPATH. To see if they're in your path, just open a command
prompt and type \texttt{java} or \texttt{javac}. If the command works, that
command is in your path. If it tells you it can't find the command, then you
need to add the folder containing either \texttt{java.exe} or
\texttt{javac.exe} to your \envPATH.

In Vista and 7, you can edit the environment variables by right clicking on
My Computer, hitting Properties, clicking "Advanced system settings" on the left
side. On the window that appears, click "Environment Variables"

TODO: finish this

\subsection{Mac OS X}
You should already have Java. You can confirm this by opening a terminal and
typing \texttt{java -version} and \texttt{javac -version}

\section{A Note on CLASSPATH}
\envCLASSPATH is an environment variable that controls where Java looks for
classes. It is a list of directories, and, like the \envPATH variable, is
separated by either a \texttt{:} --- if you use a *nix machine, such as
Linux --- or a \texttt{;} --- if you use a Windows machine.
At the very least, yours should include \texttt{.} (the current directory).
If it does not, you \textbf{must} manually append it to any commands using
the \texttt{-cp} argument. This is covered below.

\section{Getting SALSA}
Download SALSA. You can get it from \url{http://wcl.cs.rpi.edu/salsa/}.

\section{Compiling SALSA Code}
SALSA code is compiled to Java code, which is then compiled to a Java class
file. The SALSA compiler, like any compiler, is just a program. The SALSA
compiler is a Java program, so we must execute it using Java.

To run the compiler on *nix:

\texttt{java -cp \textit{location-of-salsa-jar}:. salsac.SalsaCompiler
\textit{file-to-compile}}

To run the compiler on Windows:

\texttt{java -cp \textit{location-of-salsa-jar};. salsac.SalsaCompiler
\textit{file-to-compile}}

Replace the italicized parts with the appropriate things. Note the
\texttt{-cp} -- this is short for class-path. These locations, in addition to
the \envCLASSPATH variable, are searched for classes. Since we don't want to
keep the SALSA \texttt{.jar} file in the \envCLASSPATH, we put it in the
\texttt{-cp}. Note that this works like the environment variable: Use
semicolons if you're on Windows.

\section{Compiling Java / SALSA Compiler Output}
The SALSA compiler will compile the SALSA code into Java. If this seems
strange, consider this: gcc/g++ compiles C/C++ code into assembler, before
feeding that to an assembler for translation to machine code. SALSA does
something very similar: it does SALSA $ \to $ Java $ \to $ Java bytecode.

To compile the SALSA compiler's Java output, compile it as you would any
normal java code:

\texttt{javac -cp \textit{location-of-salsa-jar}:. \textit{java-code-file}}

\section{Running the SALSA Program}
Once you've got the Java compiled, you're at your last step: running the
program. Since it's really just compiled Java, we use Java to run it.

One thing to note: Java is picky about modules. If your SALSA file starts
with:

\texttt{module foobar;}

...then it must be in a folder named "foobar". So, if your file was named
\texttt{Baz.salsa}, then you would run:

\texttt{java -cp \textit{location-of-salsa-jar}:. foorbar.Baz}

...from the \textit{parent directory} of foobar.

\section{Example}
This is our sample file:
\lstinputlisting[caption=HelloWorld.salsa,frame=tb]{HelloWorld.salsa}

Assume we have a folder \texttt{$\sim$/proglang/salsa} (yours can differ,
substitute appropriately). Inside that folder, create a \texttt{demo} folder,
and put \texttt{HelloWorld.salsa} inside that folder. Then, download the SALSA
jar file (it has a name that is along the lines of
\texttt{salsaX.X.X\_versionX.jar}), and put it inside the
\texttt{$\sim$/proglang/salsa} folder.

Open a terminal/command prompt to our \texttt{$\sim$/proglang/salsa} folder.
Do the following to compile (remember, if you're on Windows, replace
\texttt{:} with \texttt{;} in the \texttt{-cp} argument.):

\begin{lstlisting}
java -cp salsa1.1.2_version2.jar:. salsac.SalsaCompiler demo/*.salsa
javac -cp salsa1.1.2_version2.jar:. demo/*.java
java -cp salsa1.1.2_version2.jar:. demo.HelloWorld
\end{lstlisting}

\section{Common Errors}
\subsection{Could not find the main class}
You'll see an exception like:
\begin{lstlisting}
Exception in thread "main" java.lang.NoClassDefFoundError: demo/HelloWorld
Caused by: java.lang.ClassNotFoundException: demo.HelloWorld
	at ...
	at ...
	...
Could not find the main class: HelloWorld.  Program will exit.
\end{lstlisting}

Java couldn't find the class you supplied on the command line,
\texttt{demo.HelloWorld} here. If your class is HelloWorld, and it's in module
demo in the SALSA code, you must run java from the parent directory, supplying
\texttt{demo.HelloWorld} as an arguement, and there must be a
\texttt{HelloWorld.class} file in a subfolder named \texttt{demo}.

This error is as simple as Java saying, "404, not found." If you said:

\texttt{java -cp ... demo.HelloWorld}

It is going to look for \texttt{demo/HelloWorld.class}. If it can't find it,
you get this exception.
\end{document}
